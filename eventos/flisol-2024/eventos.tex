%%%%%%%%%%%%%%%%%%%%%%%%%%%%%%%%%%%%%%%%%%%%%%%%%%%%%%%%%%%%%%%%%%%%%%%%%%%%%%%%%%%%%%%%%
% Autor:        Aguilar Enriquez, Paul Sebastian a.k.a. Penserbjorne
% Autor:        @yens7
% Autor:        Calderón Olalde, Enrique Job @ksobrenat32
% Fecha:        05/01/2024
% Descripción:  Plantilla base para eventos.
%%%%%%%%%%%%%%%%%%%%%%%%%%%%%%%%%%%%%%%%%%%%%%%%%%%%%%%%%%%%%%%%%%%%%%%%%%%%%%%%%%%%%%%%%

\documentclass[a4paper,11pt]{article}                 % Papel tamaño carta, texto de 11pt.

\usepackage[top=2cm, bottom=2cm, left=2.2cm, right=2.2cm]{geometry} % Margenes
\usepackage[T1]{fontenc}                              % Indicamos la codificación de las fuentes.
\usepackage[utf8x]{inputenc}                          % Definimos la codificación.
\usepackage{lmodern}                                  % Para poder usar acentos.
\usepackage[spanish]{babel}                           % Usaremos idioma español.
\usepackage{amsmath}                                  % Para formulas matemáticas.
\usepackage{graphicx}                                 % Para imágenes.
\usepackage{float}                                    % Para posicionar objetos.
\usepackage{booktabs}                                 % Para formatear tablas.
\usepackage{hyperref}                                 % Para enlaces y referencias.


%%%%%%%%%%%%%%%%%%%%%%%%%%%%%%%%%%%%%%%%%%%%%%%%%%%%%%%%%%%%%%%%%%%%%%%%%%%%%%%%%%%%%%%%%

% Los logos tienen posiciones relativas al nombre de la escuela.
% Cada imagen esta desplazada con respecto al texto, en este caso nombre de la universidad.
% No se necesitan paquetes adicionales, el entorno estándar para imágenes de LaTeX puede hacerlo.
% El truco esta en definir una imagen de tamaño cero, asi no afecta al centrar los títulos.
\def\logoUNAM{%
  \begin{picture}(0,0)\unitlength=1cm
    \put (-3.5,-3) {\includegraphics[width=8em]{images/escudo-unam}}
  \end{picture}
}

\def\logoFI{%
  \begin{picture}(0,0)\unitlength=1cm
    \put (0.5,-3) {\includegraphics[width=8em]{images/escudo-fi}}
  \end{picture}
}

%%%%%%%%%%%%%%%%%%%%%%%%%%%%%%%%%%%%%%%%%%%%%%%%%%%%%%%%%%%%%%%%%%%%%%%%%%%%%%%%%%%%%%%%%

\author{LIDSOL}                                             % Autor de la actividad.
\title{FLISoL 2024}         % Titulo de la actividad.
\date{2024}                                             % Fecha de entrega.

\def\universidad{Universidad Nacional Autónoma de México}   % Nombre de la universidad.
\def\facultad{Facultad de Ingeniería}                       % Nombre de la facultad.
\def\semestre{2024-2}                                       % Semestre lectivo.
\def\laboratorio{Laboratorio de Investigación y Desarrollo del Software Libre}               % Nombre del laboratorio.
\makeatletter

%%%%%%%%%%%%%%%%%%%%%%%%%%%%%%%%%%%%%%%%%%%%%%%%%%%%%%%%%%%%%%%%%%%%%%%%%%%%%%%%%%%%%%%%%

\begin{document}
  % Titulo del documento con logos.
  \begin{center}
    \logoUNAM {\Large \universidad} \logoFI\par
    {\large \facultad}\par

    \laboratorio\par
    \semestre\par
    \@author\par
    \@date\par
    \@title
  \end{center}

  \hrulefill\par

  \pagenumbering{gobble}                              % Oculta el numero de pagina.
  \tableofcontents                                    % Crea el indice o tabla de contenido.

%%%%%%%%%%%%%%%%%%%%%%%%%%%%%%%%%%%%%%%%%%%%%%%%%%%%%%%%%%%%%%%%%%%%%%%%%%%%%%%%%%%%%%%%%

  \newpage
  \pagenumbering{arabic} % Muestra el numero de pagina.

  \section{Planeación}
  \subsection{Notas generales}
    La planeación de este evento tuvo inicio en marzo, teniendo avances significativos en abril. Se realizaron juntas entre los miembros activos del laboratorio. Durante este periodo se presentaron ideas de talleres y actividades a llevar a cabo durante el festival.

    Se notó que no contamos con equipos preparados para el día, ni con publicidad por lo que se acordaron actividades que se llevarían a cabo para contar con ambos y con permisos para el uso del espacio.
  \subsection{Actividades}
    \subsubsection{Instalación de software libre}
      La actividad principal del evento. Consiste en que personas lleven sus equipos de computo y se les apoye durante el proceso de instalación de una distribución de Linux, mencionando un poco sobre su funcionamiento. Las distribuciones seleccionadas a instalar en este festival fueron Fedora, Debian y Ubuntu.
    \subsubsection{Curso de introducción a arduino}
      Consiste en ner un espacio donde las personas puedan empezar a interactuar con el sistema arduino, enseñando conceptos básicos de electronica y circuitos.
    \subsubsection{Curso de introducción a git}
      Consiste en tener un espacio en el cual las personas que no conocen el sistema de control de versiones git, se vean expuestas al flujo de trabajo, sus conceptos y puedan integrarlo en sus proyectos.
    \subsubsection{Publicidad}
      Consiste en mencionar al laboratorio como un lugar donde pueden asistir personas a enterarse sobre software libre, participar en proyectos, cursos y experimentar con las distintas tecnologías libres.
  \subsection{Tabla de resumen}
    \begin{center}
    \begin{tabular}{|c | c|}
      \hline
      Actividad & Horario \\
      \hline
      Instalación de software libre & Miércoles y Jueves 10:00 - 16:00 \\
      Curso de introducción a arduino & Miércoles 13:00 - 16:00 \\
      Curso de introducción a git & Jueves 13:00 - 15:00 \\
      \hline
    \end{tabular}
    \end{center}
  \subsection{Requerimientos}
    \begin{itemize}
      \item Espacio de exposición dentro de la Facultad de Ingeniería.
      \item Mesas y sillas.
      \item Puntos de conexión eléctrica.
      \item Flyer y presentación con información del LIDSoL.
      \item USB's con distribuciones de GNU/Linux.
      \item Televisión.
    \end{itemize}

  \section{Ejecución}
  \subsection{Notas generales}
    Previo al inicio del evento se consiguieron playeras para los miembros del laboratorio que estarían presentes en el festival. Además, se consiguió una lona con el logo del laboratorio. El espacio dentro de la facultad se solicitó comunicándose con [FALTA] quien también nos proporcionó con mobiliario para la feria.

    De laboratorio se llevaron multicontactos, extensiones, se imprimieron flyers con códigos qr para dar rápido acceso a redes sociales. Se consiguieron 4 USB's con los 3 sistemas a instalar y se montó el puesto.

    Para atraer gente se tuvo la idea de contar con una botarga, uno de los miembros del laboratorio asistió con una botarga de tiburón.
  \subsection{Actividades}
    \subsubsection{Instalación de software libre}
      Varias personas presentaron interés en la instalación existiendo casos donde se acercaban con dudas sobre sistemas de virtualización y con equipos antiguos en los que no se les pudo proporcionar ayuda en el lugar por lo que se les invitó a pasar por el laboratorio. Se realizaron 3 instalaciones exitosas de sistemas operativos, 2 Ubuntu y 1 Fedora.
    \subsubsection{Curso de introducción a arduino}
      Se llevó a cabo como se esperó, llevando los equipos con los que contaba el laboratorio. Asistieron 4 interesados.
    \subsubsection{Concurso de SuperTuxKart}
      Debido a un problema se sustituyó la actividad del curso de git con un concurso. A este asistieron 4 personas en varias rondas. Esta actividad atrajo la atención de externos. Se cooperó para un premio simbólico para el ganador.
    \subsubsection{Publicidad}
      Multiples personas de distintos ámbitos se acercaron al laboratorio preguntando sobre proyectos, ubicación, como pertenecer e información general sobre el software libre. Se les proporcionaron las redes sociales y mencionaron proyectos y cursos actuales.

  \section{Retroalimentación}
  \subsection{Notas generales}
    El primer día nos posicionamos en una esquina lo que limitó la visibilidad y notamos que el segundo día estando en la entrada de la facultad mucha mas gente se acercó.

    Tardamos en empezar el proceso para pedir el espacio lo que dificultó el mismo. Debemos de solicitar el espacio con mucho mas tiempo de anticipación

    Nos hizo falta publicidad, mucha gente no conoce el laboratorio y no estaba enterada del evento. Mas publicidad tanto física como digital previo al evento puede ser de gran ayuda.

    Las mesas proporcionadas no contaban con mantel, aunque no estrictamente necesario, estéticamente es recomendado.

%%%%%%%%%%%%%%%%%%%%%%%%%%%%%%%%%%%%%%%%%%%%%%%%%%%%%%%%%%%%%%%%%%%%%%%%%%%%%%%%%%%%%%%%%
\end{document}
