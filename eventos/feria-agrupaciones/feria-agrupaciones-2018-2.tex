%%%%%%%%%%%%%%%%%%%%%%%%%%%%%%%%%%%%%%%%%%%%%%%%%%%%%%%%%%%%%%%%%%%%%%%%%%%%%%%%%%%%%%%%%
% Autor:        Aguilar Enriquez, Paul Sebastian a.k.a. Penserbjorne
% Fecha:        05/02/2017
% Descripcion:  Plantilla base para actividades o tareas.
%%%%%%%%%%%%%%%%%%%%%%%%%%%%%%%%%%%%%%%%%%%%%%%%%%%%%%%%%%%%%%%%%%%%%%%%%%%%%%%%%%%%%%%%%

\documentclass[a4paper,11pt]{article}                 % Papel tamaño carta, texto de 11pt.

\usepackage[top=2cm, bottom=2cm, left=2.2cm, right=2.2cm]{geometry} % Margenes
\usepackage[T1]{fontenc}                              % Indicamos la codificacion de las fuentes.
\usepackage[utf8x]{inputenc}                          % Definimos la codificacion.
\usepackage{lmodern}                                  % Para poder usar acentos.
\usepackage[spanish]{babel}                           % Usaremos idioma español.
\usepackage{amsmath}                                  % Para formulas matematicas.
\usepackage{graphicx}                                 % Para imagenes.
\usepackage{float}                                    % Para posicionar objetos.
\usepackage{booktabs}                                 % Para formatear tablas.
\usepackage{hyperref}                                 % Para enlaces y referencias.
\usepackage{lscape}
 \usepackage{timetable}
% \usepackage[spanish.mexico]{babel}
\usepackage{tabularx}


%%%%%%%%%%%%%%%%%%%%%%%%%%%%%%%%%%%%%%%%%%%%%%%%%%%%%%%%%%%%%%%%%%%%%%%%%%%%%%%%%%%%%%%%%

% Los logos tienen posiciones relativas al nombre de la escuela.
% Cada imagen esta desplazada con respecto al texto, en este caso nombre de la univseridad.
% No se necesitan paquetes adicionales, el entorno estandar para imagenes de LaTeX puede hacerlo.
% El truco esta en definir una imagen de tamaño cero, asi no afecta al centrar los titulos.
\def\logoUNAM{%
  \begin{picture}(0,0)\unitlength=1cm
    \put (-3.5,-3) {\includegraphics[width=8em]{images/escudo-unam}}
  \end{picture}
}

\def\logoFI{%
  \begin{picture}(0,0)\unitlength=1cm
    \put (0.5,-3) {\includegraphics[width=8em]{images/escudo-fi}}
  \end{picture}
}

%%%%%%%%%%%%%%%%%%%%%%%%%%%%%%%%%%%%%%%%%%%%%%%%%%%%%%%%%%%%%%%%%%%%%%%%%%%%%%%%%%%%%%%%%

\author{LIDSOL}  % Autor de la actividad.
\title{Lista de actividades \\ Feria de Agrupaciones Estudiantiles}                % Titulo de la actividad.
\date{03/04/2018}                                           % Fecha de entrega.
\def\universidad{Universidad Nacional Autónoma de México}   % Nombre de la universidad.
\def\facultad{Facultad de Ingeniería}                              % Nombre de la facultdad.
\def\semestre{2018-2}                                     % Semestre lectivo.
\def\materia{Laboratorio de Investigación y Desarrollo del Software Libre}               % Nombre de la materia y grupo.
\makeatletter

%%%%%%%%%%%%%%%%%%%%%%%%%%%%%%%%%%%%%%%%%%%%%%%%%%%%%%%%%%%%%%%%%%%%%%%%%%%%%%%%%%%%%%%%%

\begin{document}
  
  % Titulo del documento con logos.
  \begin{center}
    \logoUNAM {\Large \universidad} \logoFI\par
    {\large \facultad}\par

    \materia\par
    \semestre\par
   % \@author\par
    \@date\par
    \@title
  \end{center}

  \hrulefill\par

  \pagenumbering{gobble}                              % Oculta el numero de pagina.
  \tableofcontents                                    % Crea el indice o tabla de contenido.

%%%%%%%%%%%%%%%%%%%%%%%%%%%%%%%%%%%%%%%%%%%%%%%%%%%%%%%%%%%%%%%%%%%%%%%%%%%%%%%%%%%%%%%%%

  \newpage                                            % Inserta una pagina nueva.

  \begin{landscape}



  \section*{Tabla de resumen}
 
  
  \begin{table}[H]
\centering
%\caption{My caption}

\begin{tabular}{|l|l|l|l|l|l|}
\hline
 Actividad & Elementos & Duración  & Ponente & Lugar & Fecha y hora \\
   & necesarios & (\#días) &  &  &  \\ \hline 
 Proyección de OpenMovies & Ver lista ~\ref{list:openmovies}  & 2 hrs (1) & Miembros de LIDSOL & Lab. de iOS, Edif. P &  Mi 18 Abril, 17:00-19:00 hrs \\ \hline 
 Installfest & Ver lista ~\ref{list:installfest}  & 3,5 hrs (1)  & Miembros de LIDSOL & Stands & Ju 19 Abril, 10:00-13:00 hrs\\ 
   &  & 6,5 hrs (1)&  & o lugar que se hablilite & 18 y 19 Abril, 15:30-19:00 hrs\\ \hline 
 Taller de OpenScad & Ver lista ~\ref{list:openscad} & 3 hrs (2) & Pablo Vivar & Sala Microsoft Research &  Mi 18 Abril, 17:00-20:00 hrs \\ 
  &  & &  & Edif. Q, 2do piso & Ju 19 Abril, 12:00-15:00 hrs\\ \hline
Taller ... impresión 3D & Ver lista ~\ref{list:impresion} & 2 hrs (1) & Emilio Cabrera & Sala Microsoft Research & Mi 18 Abril, 13:00-15:00 hrs \\ \hline
 Taller de KiCad & Ver lista ~\ref{list:kicad}  & 2 hrs (2) & Yesica Navarro & Sala Microsoft Research & Mi 18 Abril, 15:00-17:00 hrs  \\
    &  & &  & Edif. Q, 2do piso & Ju 19 Abril, 17:00-19:00 hrs\\  \hline
 Taller de Nightly & Ver lista ~\ref{list:nightly} & 2 hrs (1) & Paul Aguilar & Sala Microsoft Research & Ju 19 Abril, 15:00-17:00 hrs \\ \hline
C/P Privacidad, anonimato & Ver lista ~\ref{list:ddigitales} & 1,5 hrs (1) & Gunnar Wolf & Auditorio Sotero Prieto & Ju 19 Abril, 16:00-17:30 hrs \\ 
  y derechos digitales  &  & &  Diego Barriga &  & \\  \hline
C/P Github & Ver lista ~\ref{list:github} & 1,5 hrs (1) & Pablo Flores & Auditorio Sotero Prieto & Ju 19 Abril, 13:00-14:30 hrs \\ \hline
C/P Software Libre & Ver lista ~\ref{list:sl} & 1,5 hrs (1) & Miembros de LIDSOL & Auditorio Sotero Prieto & Mi 18 Abril, 10:00-11:30 hrs \\  \hline
\end{tabular}
 
\end{table}
  
  
  \addcontentsline{toc}{section}{Tabla de resumen}
  \end{landscape}
  
  \pagenumbering{arabic} 
                                 % Muestra el numero de pagina.
  \section{Proyección de OpenMovies.}                                     % Insertamos nueva seccion, SI aparece en la tabla de contenido.
  
  Esta actividad consiste en la proyección de OpenMovies durante la feria, antes y después de cada proyección se explicará cuál es la filosofía detrás de este tipo de películas, qué herramientas se utilizan y su proceso de producción.
  \paragraph{}
 \textbf{Elementos necesarios para llevar a cabo esta actividad.}
  \begin{itemize}
    \label{list:openmovies}
    \item Cañon.
    \item Bocinas.
    \item Espacio para proyección.
    \item Asientos para los asistentes.
  \end{itemize}
  
  \textbf{Duración.}
  \begin{itemize}
    \item Las películas duran entre 15 y 45 minutos.
    \item Se esperan proyectar 2 horas un día. Listado en 
    \url{https://goo.gl/6Zu1Fn}
  \end{itemize}
  
  
    \section{Installfest.}                                     % Insertamos nueva seccion, SI aparece en la tabla de contenido.
    Esta actividad consiste en promover el uso e instalación de distribuciones GNU Linux para uso personal y académico. Se asesorará de acuerdo a las necesidades de cada persona cuál es la distribución que más se adecua e ella.
  \paragraph{}
   \textbf{Elementos necesarios para llevar a cabo esta actividad.}
  \begin{itemize}
    \label{list:installfest}
    \item Conexión a internet.
    \item Puntos de conexión a la red eléctrica (tomacorrientes).
    \item Mesas.
    \item USB's de 2 GB a 4 GB.
  \end{itemize}
  
  \textbf{Duración.}
  \begin{itemize}
    \item 10 horas. El primer día con un bloque de 3,5 hrs y el segundo día con dos bloques, uno de 3 hrs y el otro de 3,5 hrs.
  \end{itemize}
  
      \section{Taller de OpenScad. " {OpenScad} para diseño de módelos parametrizables 2D y 3D ".}                                     % Insertamos nueva seccion, SI aparece en la tabla de contenido.

  Este taller consiste en la presentación de OpenScad para el modelado parametrizable 2D y 3D, el cuál sirve para la elaboración de planos que puedan ser manufacturados en máquinas de diseño (cortadora láser e impresora 3D).
  Temario: \url{https://lidsol.net/talleres/0003_openscad_basico.html}
      \paragraph{}
  \textbf{Elementos necesarios para llevar a cabo esta actividad.}
  \begin{itemize}
    \label{list:openscad}
    \item Conexión a internet.
    \item Espacio para el taller con equipo de cómputo. 
    \item Ver requerimientos de software en página~\pageref{list:openscads}.
  \end{itemize}
  
  \textbf{Duración.}
  \begin{itemize}
    \item 3 horas, 2 días.
  \end{itemize}
  
            \textbf{Ponente.}
  \begin{itemize}
    \item Pablo Vivar.
  \end{itemize}
  
  
  \section{Taller de monitoreo y administración de una impresora 3D. " {Administra} tu impresora 3D en línea ".}                                     % Insertamos nueva seccion, SI aparece en la tabla de contenido.

   En esta actividad se pretende mostrar todo lo que se necesita para monitorear y administrar una impresora 3D.
      \paragraph{}
  \textbf{Elementos necesarios para llevar a cabo esta actividad.}
  \begin{itemize}
    \label{list:impresion}
    \item Conexión a internet.
    \item Espacio para el taller con equipo de cómputo.
    \item Ver requerimientos de software en página~\pageref{list:impresions}.
  \end{itemize}
  
  \textbf{Duración.}
  \begin{itemize}
    \item 2 horas, un día.
  \end{itemize}
  
              \textbf{Ponente.}
  \begin{itemize}
    \item Emilio Cabrera.
  \end{itemize}
  
  
              \section{Taller de KiCad. " Tu primer PCB con KiCad ".}                                     % Insertamos nueva seccion, SI aparece en la tabla de contenido.

   Este taller consiste en la presentación de KiCad para hacer placas PCB de circuitos electrónicos.
      \paragraph{}
  \textbf{Elementos necesarios para llevar a cabo esta actividad.}
  \begin{itemize}
  \label{list:kicad}
    \item Conexión a internet.
    \item Espacio para el taller con equipo de cómputo.
    \item Ver requerimientos de software en página~\pageref{list:kicads}.
  \end{itemize}
  
  \textbf{Duración.}
  \begin{itemize}
    \item 2 horas, 2 días.
  \end{itemize}
  
              \textbf{Ponente.}
  \begin{itemize}
    \item Yesica Navarro.
  \end{itemize}
  
  
                \section{Taller de Nightly. " {Cómo} contribuir a Firefox sin saber programación ".}                                     % Insertamos nueva seccion, SI aparece en la tabla de contenido.

   En este taller se mostrará como instalar y configurar Firefox Nightly, se explicará la importancia de contribuir con pruebas en un software en etapa beta, cómo probarlo y reportar bugs. 
      \paragraph{}
  \textbf{Elementos necesarios para llevar a cabo esta actividad.}
  \begin{itemize}
    \label{list:nightly}
    \item Conexión a internet.
    \item Espacio para el taller con equipo de cómputo.
    \item Ver requerimientos de software en página~\pageref{list:nightlys}.
  \end{itemize}
  
  \textbf{Duración.}
  \begin{itemize}
    \item 2 horas, un día.
  \end{itemize}
  
              \textbf{Ponente.}
  \begin{itemize}
    \item Paul Aguilar.
  \end{itemize}
  
  \vspace{1 cm}
  
            \section{Conferencia/Plática " Privacidad, anonimato y derechos digitales ".}                                     % Insertamos nueva seccion, SI aparece en la tabla de contenido.

   En esta conferencia se abordará el contexto actual de los derechos digitales, los riesgos y amenazas que existen en torno a ellos, después se procedera a hablar sobre mecánismos de privacidad y anonimato en la red.
      \paragraph{}
  \textbf{Elementos necesarios para llevar a cabo esta actividad.}
  \begin{itemize}
    \label{list:ddigitales}
    \item Auditorio Sotero Prieto.
    \item Conexión a internet. (Transmisión en vivo)
  \end{itemize}
  
  \textbf{Duración.}
  \begin{itemize}
    \item 1,5 horas.
  \end{itemize}
  
    \textbf{Ponente.}
  \begin{itemize}
    \item Gunnar Wolf.
    \item Diego Barriga.
  \end{itemize}
  

  
  \section{Conferencia/Plática " ¿Hiciste cambios y ya no compila? Hablemos de Git ".}                                     % Insertamos nueva seccion, SI aparece en la tabla de contenido.
   En esta plática se hablará de la importancia de utilizar un control de versiones para proyectos universitarios y su contribución a la supervivencia de proyectos libres.
      \paragraph{}
  \textbf{Elementos necesarios para llevar a cabo esta actividad.}
  \begin{itemize}
    \label{list:github}
    \item Auditorio Sotero Prieto.
        \item Conexión a internet. (Transmisión en vivo)
  \end{itemize}
  
  \textbf{Duración.}
  \begin{itemize}
    \item 1,5 horas.
  \end{itemize}
  
        \textbf{Ponente.}
  \begin{itemize}
    \item Pablo Flores.
  \end{itemize}
  
    \vspace{1 cm}
  

  
      \section{Conferencia/Plática " No es tu amigo, es software privativo ".}                                     % Insertamos nueva seccion, SI aparece en la tabla de contenido.

   En esta plática se pretende hablar de qué es el FOSS y su impacto en el mundo. 
      \paragraph{}
  \textbf{Elementos necesarios para llevar a cabo esta actividad.}
  \begin{itemize}
    \label{list:sl}
    \item Auditorio Sotero Prieto.
        \item Conexión a internet. (Transmisión en vivo)
  \end{itemize}
  
  \textbf{Duración.}
  \begin{itemize}
    \item 1,5 horas.
  \end{itemize}
  
            \textbf{Ponentes.}
  \begin{itemize}
    \item Paul Aguilar.
    \item Pablo Vivar.
    \item Emilio Cabrera.
    \item Diego Barriga.
    \item Yesica Navarro.
  \end{itemize}
  
  \thispagestyle{empty}
 \begin{landscape}
 %\printheading{Horarios de feria de agrupaciones.}
 
 % Define the layout of your time tables
 \setslotsize{2.8cm}{0.25cm}
 % (columnas de días), 
 \setslotcount{6}{50}
 \settopheight{3}
 \settextframe{1.0mm}
 \setframetype[c]{1}
 
 % Define event types
 %            type         r     g     b     t_r  t_g  t_b
 
  %Yesica Navaroo
 \defineevent{YES}{0.98} {0.15}{0.44} {1.0}{1.0}{1.0}
 
 %Emilio Cabrera
  \defineevent{EM}{0.4} {0.85} {0.94} {1.0}{1.0}{1.0}
 
 %Luis Vilchis
 \defineevent{LUIS}{0.65}{0.89} {0.18}{1.0}{1.0}{1.0}

%Pablo Vivar
 \defineevent{PABS}{0.77}{0.55}{1.0}{1.0}{1.0}{1.0}
 
 %Marco Ruano
 \defineevent{MARCO}{0.47}{0.37}{1.0}{1.0}{1.0}{1.0}
 
 %Pablo Flores
 \defineevent{PABLOF}{0.5}{0.35}{0.45}{1.0}{1.0}{1.0}
 
 %Pablo Vivar  Openscad
 \defineevent{OpSCAD}{0.93}{0.89}{0.49}{0.45}{0.45}{0.45}
 
 %Emilio Cabrera Impresion 3D
  \defineevent{Imp3D} {0.1}{0.67}{1.0}{1.0}{1.0}{1.0}   
 
 %Yesica Navarro Kikad
 \defineevent{KiCAD}{0.65}{0.89} {0.18}{1.0}{1.0}{1.0}

%Sebastian Aguilar Nightly
 \defineevent{Night}{1.0}{0.65}{0.0}{1.0}{1.0}{1.0}
 
 %LIDSOL
  \defineevent{COLID} {0.98} {0.15}{0.44} {1.0}{1.0}{1.0}
  
  %
 \defineevent{OpMov}{0.77}{0.55}{1.0}{1.0}{1.0}{1.0}
 
%Sebastian Aguilar
 \defineevent{PULK}{0.13}{0.73}{0.1}{0.25}{0.25}{0.25}
 
 %Emilio Cabrera
  \defineevent{EM} {0.1}{0.67}{1.0}{1.0}{1.0}{1.0}   
 
 %Luis Vilchis
 \defineevent{LUIS}       {0.65}{0.89} {0.18}{1.0}{1.0}{1.0}

%Pablo Vivar
 \defineevent{PABS} 
 {0.4} {0.85} {0.94} {1.0}{1.0}{1.0}
 
  
 %Install fest
 \defineevent{InsFest}{0.1}{0.35}{0.75}{1.0}{1.0}{1.0}
 
\section*{Horario de Feria de Agrupaciones Estudiantiles.}
 % Start the time table
 \begin{timetable}
 
   \hours{10}{12}{1}

   %\frenchdays{1}
% \spanishdays{1}
   \daymark{}
 \daymark{ Mi 18 de Abril  }
  \daymark{}
     \daymark{$ \mapsto$ }
 \daymark{Ju 19 de Abril }
  \daymark{ }
   %      x start  end    name                                     lecturer          location              type
   
  
   
   %####MIERCOLES####
   
  \event 1 {1000}{1200}{Pl\'{a}tica - No es tu amigo, es software privativo}{LIDSOL}{{\tiny LIDSOL}}{COLID}
  
   \event 1 {1300}{1500}{Taller Impresi\'{o}n 3D}{Emilio Cabrera}{{\tiny LIDSOL}}{Imp3D}
  
    \event 1 {1500}{1700}{Taller KiCAD}{Yesica Navarro}{{\tiny LIDSOL}}{KiCAD}
    
    \event 1 {1700}{2000}{Taller OpenSCAD}{Pablo Vivar Colina}{{\tiny LIDSOL}}{OpSCAD}
    
    
  % \event 2 {1300}{1430}{ Combatiendo 1984 en el 2018 con Mozilla }{Uriel Jurado}{{\tiny Paul y Yesica}}{Night}
 
  %\event 2 {1400}{1530}{Paul invita comida}{Mozilla}{{\tiny Paul}}{PULK}
  
  \event 2 {1700}{1900}{Open Movies}{LIDSOL}{{\tiny Paul}}{OpMov}
   
   
   \event 3 {1530}{1900}{Install Fest}{LIDSOL}{{\tiny Di, Em. y Luis}}{InsFest}
   
  
  
  
   %######JUEVES####
   
   \event 4 {1200}{1500}{Taller OpenSCAD}{Pablo Vivar Colina}{{\tiny LIDSOL}}{OpSCAD}
   
    \event 4 {1500}{1700}{Taller Nightly}{Sebasti\'{a}n Aguilar}{{\tiny LIDSOL}}{Night}
    
    \event 4 {1700}{1900}{Taller KiCAD}{Yesica Navarro}{{\tiny LIDSOL}}{KiCAD}
    
    

    
    \event 5 {1300}{1430}{Pl\'{a}tica - GitHub }{Pablo Flores}{{\tiny Yesica}}{PABLOF}
   
   %\event 5 {1430}{1530}{GitHub}{Comida Yesica}{{\tiny Yesica}}{YES}
   
   \event 5 {1600}{1730}{Pl\'{a}tica - DDD y MP }{Wolf y Diego}{{\tiny Pablo}}{PABS}
 
% \event 5 {1730}{1830}{Wolf}{Diego}{{\tiny Diego}}{YES}
 
 \event 6 {1000}{1300}{Install Fest}{LIDSOL}{{\tiny Emilio}}{InsFest}
   
   \event 6 {1530}{1900}{Install Fest}{LIDSOL}{{\tiny Emilio y Luis}}{InsFest}
   
   
   
   
 \end{timetable}
 
 
   \addcontentsline{toc}{section}{Horario de Feria de Agrupaciones.}
 \end{landscape}
 
  \section*{Lista de requerimientos de software para los talleres.}
   \addcontentsline{toc}{section}{Lista de requerimientos de software para los talleres.}
   
         \subsection*{Taller de OpenScad.}                                     % Insertamos nueva seccion, SI aparece en la tabla de contenido.
         \addcontentsline{toc}{subsection}{Taller de OpenScad.}
  \label{list:openscads}
  \begin{itemize}
    \item OpenSCAD 2015.
  \end{itemize}
  
        \subsection*{Taller de monitoreo y administración de una impresora 3D.}                                     % Insertamos nueva seccion, SI aparece en la tabla de contenido.
        \addcontentsline{toc}{subsection}{Taller de monitoreo y administración de una impresora 3D.}
  \label{list:impresions}
  \begin{itemize}
    \item Python.
    \item Haproxy.
    \item OpenVPN.
    \item Curaengine.
  \end{itemize}
  
        \subsection*{Taller de KiCad.}                                     % Insertamos nueva seccion, SI aparece en la tabla de contenido.
        \addcontentsline{toc}{subsection}{Taller de KiCad.}
  \label{list:kicads}
\begin{itemize}
    \item KiCad 4.0.7.
  \end{itemize}
                \subsection*{Taller de Nightly.}                                     % Insertamos nueva seccion, SI aparece en la tabla de contenido.
                \addcontentsline{toc}{subsection}{Taller de Nightly.}
  \label{list:nightlys}
  \begin{itemize}
    \item Nightly 59.0a1.
  \end{itemize}
  
   
%%%%%%%%%%%%%%%%%%%%%%%%%%%%%%%%%%%%%%%%%%%%%%%%%%%%%%%%%%%%%%%%%%%%%%%%%%%%%%%%%%%%%%%%%
  


%%%%%%%%%%%%%%%%%%%%%%%%%%%%%%%%%%%%%%%%%%%%%%%%%%%%%%%%%%%%%%%%%%%%%%%%%%%%%%%%%%%%%%%%%

\end{document}
