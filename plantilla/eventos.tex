%%%%%%%%%%%%%%%%%%%%%%%%%%%%%%%%%%%%%%%%%%%%%%%%%%%%%%%%%%%%%%%%%%%%%%%%%%%%%%%%%%%%%%%%%
% Autor:        Aguilar Enriquez, Paul Sebastian a.k.a. Penserbjorne
% Autor:        @yens7
% Autor:        Calderón Olalde, Enrique Job @ksobrenat32
% Fecha:        05/01/2024
% Descripción:  Plantilla base para eventos.
%%%%%%%%%%%%%%%%%%%%%%%%%%%%%%%%%%%%%%%%%%%%%%%%%%%%%%%%%%%%%%%%%%%%%%%%%%%%%%%%%%%%%%%%%

\documentclass[a4paper,11pt]{article}                 % Papel tamaño carta, texto de 11pt.

\usepackage[top=2cm, bottom=2cm, left=2.2cm, right=2.2cm]{geometry} % Margenes
\usepackage[T1]{fontenc}                              % Indicamos la codificación de las fuentes.
\usepackage[utf8x]{inputenc}                          % Definimos la codificación.
\usepackage{lmodern}                                  % Para poder usar acentos.
\usepackage[spanish]{babel}                           % Usaremos idioma español.
\usepackage{amsmath}                                  % Para formulas matemáticas.
\usepackage{graphicx}                                 % Para imágenes.
\usepackage{float}                                    % Para posicionar objetos.
\usepackage{booktabs}                                 % Para formatear tablas.
\usepackage{hyperref}                                 % Para enlaces y referencias.


%%%%%%%%%%%%%%%%%%%%%%%%%%%%%%%%%%%%%%%%%%%%%%%%%%%%%%%%%%%%%%%%%%%%%%%%%%%%%%%%%%%%%%%%%

% Los logos tienen posiciones relativas al nombre de la escuela.
% Cada imagen esta desplazada con respecto al texto, en este caso nombre de la universidad.
% No se necesitan paquetes adicionales, el entorno estándar para imágenes de LaTeX puede hacerlo.
% El truco esta en definir una imagen de tamaño cero, asi no afecta al centrar los títulos.
\def\logoUNAM{%
  \begin{picture}(0,0)\unitlength=1cm
    \put (-3.5,-3) {\includegraphics[width=8em]{images/escudo-unam}}
  \end{picture}
}

\def\logoFI{%
  \begin{picture}(0,0)\unitlength=1cm
    \put (0.5,-3) {\includegraphics[width=8em]{images/escudo-fi}}
  \end{picture}
}

%%%%%%%%%%%%%%%%%%%%%%%%%%%%%%%%%%%%%%%%%%%%%%%%%%%%%%%%%%%%%%%%%%%%%%%%%%%%%%%%%%%%%%%%%

\author{LIDSOL}                                             % Autor de la actividad.
\title{Nombre de actividad~\footnote{Aclaraciones}}         % Titulo de la actividad.
\date{01/01/00}                                             % Fecha de entrega.

\def\universidad{Universidad Nacional Autónoma de México}   % Nombre de la universidad.
\def\facultad{Facultad de Ingeniería}                       % Nombre de la facultad.
\def\semestre{2024-2}                                       % Semestre lectivo.
\def\laboratorio{Laboratorio de Investigación y Desarrollo del Software Libre}               % Nombre del laboratorio.
\makeatletter

%%%%%%%%%%%%%%%%%%%%%%%%%%%%%%%%%%%%%%%%%%%%%%%%%%%%%%%%%%%%%%%%%%%%%%%%%%%%%%%%%%%%%%%%%

\begin{document}
  % Titulo del documento con logos.
  \begin{center}
    \logoUNAM {\Large \universidad} \logoFI\par
    {\large \facultad}\par

    \laboratorio\par
    \semestre\par
    \@author\par
    \@date\par
    \@title
  \end{center}

  \hrulefill\par

  \pagenumbering{gobble}                              % Oculta el numero de pagina.
  \tableofcontents                                    % Crea el indice o tabla de contenido.

%%%%%%%%%%%%%%%%%%%%%%%%%%%%%%%%%%%%%%%%%%%%%%%%%%%%%%%%%%%%%%%%%%%%%%%%%%%%%%%%%%%%%%%%%

  \newpage
  \pagenumbering{arabic} % Muestra el numero de pagina.

  \section{Planeación}
  \subsection{Notas generales}
    Incluir notas generales sobre la planeación del evento.
  \subsection{Actividad n}
    Detallar cada actividad planificada para el evento.
  \subsection{Tabla de resumen}
    Proporcionar una tabla de resumen con información clave sobre el evento.
  \subsection{Horario}
    Establecer un horario detallado para el evento.
  \subsection{Requerimientos}
    Enumerar los requerimientos necesarios para llevar a cabo el evento, como equipos, materiales, etc.

  \section{Ejecución}
  \subsection{Notas generales}
    Incluir notas generales sobre la ejecución del evento.
  \subsection{Actividad n}
    Detallar cómo se llevó a cabo cada actividad planificada.

  \section{Retroalimentación}
  \subsection{Notas generales}
    Incluir notas generales sobre la retroalimentación del evento.
  \subsection{Actividad N}
    Recopilar la retroalimentación específica sobre cada actividad del evento.

%%%%%%%%%%%%%%%%%%%%%%%%%%%%%%%%%%%%%%%%%%%%%%%%%%%%%%%%%%%%%%%%%%%%%%%%%%%%%%%%%%%%%%%%%
\end{document}
